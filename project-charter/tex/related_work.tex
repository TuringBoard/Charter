
% Discuss the state-of-the-art with respect to your product. What solutions currently exist, and in what form (academic research, enthusiast prototype, commercially available, etc.)? Include references and citations as necessary using the \textit{cite} command, like this \cite{Rubin2012}. If there are existing solutions, why won't they work for your customer (too expensive, not fast enough, not reliable enough, etc.). This section should occupy 1/2 - 1 full page, and should include at least 5 references to related work. All references should be added to the \textit{.bib} file, fully documented in IEEE format, and should appear in the \textit{references} section at the end of this document (the IEEE citation style will automatically be applied if your reference is properly added to the \textit{.bib} file).

% ProTip: Consider using a citation manager such as Mendeley, Zotero, or EndNote to generate your \textit{.bib} file and maintain documentation references throughout the life cycle of the project.

At the time of the document being published there is no automated longboard commercially available. All longboards, at the moment, are either manual or electrical. Electrical longboards fix issues that come with a manual longboards. An example would be the need to constantly push the longboard in order for it to move. This is fixed by putting motors on the board and using them to push the board instead. Commercially, there are various electrical boards available\cite{Petrovan2021}. Manual longboards simplify the process of traveling on a longboard. Electrical longboards automate the "push to move" action. Our product will add on to this feature. By introducing this product, we would be walking into uncharted territories. 

An electric board that comes close to what we are trying to accomplish is the Audi's longboard that comes with the purchase of one of their vehicles. The Audi longboard offers one of the features our longboard will offer: the ability to follow a user\cite{Fox2016}. Their longboard can also be configured to be used as a scooter. As nice as their board is, it only offers the "follow user" feature. The board itself is extremely expensive because it comes with the purchase of a car.

Another commercially available product that has features similar to what our boards will offer is the Ovis suitcase. This suitcase has the ability to lock onto a user and follow them around while avoiding obstacles\cite{Henry2018}. The Ovis suitcase uses computer vision to navigate just like our board will.

%%%%%%%%%%%%%%%%%%%%%%%%
% To cite something, use the \cite command with the name of the bibtex entry in the curly braces.
% It will determine which reference number it is and insert that number where the \cite command is.
% e.g. \cite{Rubin2012}
