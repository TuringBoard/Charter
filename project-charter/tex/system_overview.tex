The Turing board will have two primary modes of operation: rider mode and autonomous mode. While in rider mode the electric long-board will be controlled entirely by the rider who is responsible for speed control. In autonomous mode the board will have the capability of being summoned from a parked location and self navigating to the user. Following the user is another function that will only be available while the electric long board is autonomous. \\
\\
Rider Mode \\
The user will provide acceleration and deceleration commands through a mobile app connected to the hardware on the board with Internet of Things (IoT) software. An electronic speed controller will be used to receive the commands from the rider and adjust the speed of the electric motor accordingly. There will be no automation while in rider mode.\\
\centerline
{
***************************\\
*IMAGE TO BE INCLUDED HERE*\\
***************************\\
}
\\
\\
Autonomous Mode \\
While in autonomous mode the commands for acceleration and deceleration will come from software analyzing real time environment data from board mounted cameras and sensors. An additional force sensor will be placed on the long board to ensure autonomous mode is only functional with a limited weight load on the board. \\
Turning without a rider present will require a unique hardware solution. During rider mode the user creates a bias in the wheel direction by leaning on either side of the board, this movement compresses the bushing component of long board trucks and allows the board to change direction. To solve this issue during autonomous mode a new hardware component will be created that will allow a stepper motor to spin the front trucks independent of the boards direction. The mechanism will fit between the truck and the board, acting as a conventional riser that separates the trucks from the deck. Using a stepper motor to drive a bevel gear, the lower component of the mechanism will spin along a vertical axis no more than thirty degrees in either direction. Solenoid pins will be used to lock the mechanism in place when not in use or when the user has switched to rider mode. \\
\centerline
{
***************************\\
*IMAGE TO BE INCLUDED HERE*\\
***************************\\
} 
